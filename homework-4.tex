% Options for packages loaded elsewhere
\PassOptionsToPackage{unicode}{hyperref}
\PassOptionsToPackage{hyphens}{url}
\PassOptionsToPackage{dvipsnames,svgnames,x11names}{xcolor}
%
\documentclass[
  letterpaper,
  DIV=11,
  numbers=noendperiod]{scrartcl}

\usepackage{amsmath,amssymb}
\usepackage{iftex}
\ifPDFTeX
  \usepackage[T1]{fontenc}
  \usepackage[utf8]{inputenc}
  \usepackage{textcomp} % provide euro and other symbols
\else % if luatex or xetex
  \usepackage{unicode-math}
  \defaultfontfeatures{Scale=MatchLowercase}
  \defaultfontfeatures[\rmfamily]{Ligatures=TeX,Scale=1}
\fi
\usepackage{lmodern}
\ifPDFTeX\else  
    % xetex/luatex font selection
\fi
% Use upquote if available, for straight quotes in verbatim environments
\IfFileExists{upquote.sty}{\usepackage{upquote}}{}
\IfFileExists{microtype.sty}{% use microtype if available
  \usepackage[]{microtype}
  \UseMicrotypeSet[protrusion]{basicmath} % disable protrusion for tt fonts
}{}
\makeatletter
\@ifundefined{KOMAClassName}{% if non-KOMA class
  \IfFileExists{parskip.sty}{%
    \usepackage{parskip}
  }{% else
    \setlength{\parindent}{0pt}
    \setlength{\parskip}{6pt plus 2pt minus 1pt}}
}{% if KOMA class
  \KOMAoptions{parskip=half}}
\makeatother
\usepackage{xcolor}
\setlength{\emergencystretch}{3em} % prevent overfull lines
\setcounter{secnumdepth}{-\maxdimen} % remove section numbering
% Make \paragraph and \subparagraph free-standing
\ifx\paragraph\undefined\else
  \let\oldparagraph\paragraph
  \renewcommand{\paragraph}[1]{\oldparagraph{#1}\mbox{}}
\fi
\ifx\subparagraph\undefined\else
  \let\oldsubparagraph\subparagraph
  \renewcommand{\subparagraph}[1]{\oldsubparagraph{#1}\mbox{}}
\fi

\usepackage{color}
\usepackage{fancyvrb}
\newcommand{\VerbBar}{|}
\newcommand{\VERB}{\Verb[commandchars=\\\{\}]}
\DefineVerbatimEnvironment{Highlighting}{Verbatim}{commandchars=\\\{\}}
% Add ',fontsize=\small' for more characters per line
\usepackage{framed}
\definecolor{shadecolor}{RGB}{241,243,245}
\newenvironment{Shaded}{\begin{snugshade}}{\end{snugshade}}
\newcommand{\AlertTok}[1]{\textcolor[rgb]{0.68,0.00,0.00}{#1}}
\newcommand{\AnnotationTok}[1]{\textcolor[rgb]{0.37,0.37,0.37}{#1}}
\newcommand{\AttributeTok}[1]{\textcolor[rgb]{0.40,0.45,0.13}{#1}}
\newcommand{\BaseNTok}[1]{\textcolor[rgb]{0.68,0.00,0.00}{#1}}
\newcommand{\BuiltInTok}[1]{\textcolor[rgb]{0.00,0.23,0.31}{#1}}
\newcommand{\CharTok}[1]{\textcolor[rgb]{0.13,0.47,0.30}{#1}}
\newcommand{\CommentTok}[1]{\textcolor[rgb]{0.37,0.37,0.37}{#1}}
\newcommand{\CommentVarTok}[1]{\textcolor[rgb]{0.37,0.37,0.37}{\textit{#1}}}
\newcommand{\ConstantTok}[1]{\textcolor[rgb]{0.56,0.35,0.01}{#1}}
\newcommand{\ControlFlowTok}[1]{\textcolor[rgb]{0.00,0.23,0.31}{#1}}
\newcommand{\DataTypeTok}[1]{\textcolor[rgb]{0.68,0.00,0.00}{#1}}
\newcommand{\DecValTok}[1]{\textcolor[rgb]{0.68,0.00,0.00}{#1}}
\newcommand{\DocumentationTok}[1]{\textcolor[rgb]{0.37,0.37,0.37}{\textit{#1}}}
\newcommand{\ErrorTok}[1]{\textcolor[rgb]{0.68,0.00,0.00}{#1}}
\newcommand{\ExtensionTok}[1]{\textcolor[rgb]{0.00,0.23,0.31}{#1}}
\newcommand{\FloatTok}[1]{\textcolor[rgb]{0.68,0.00,0.00}{#1}}
\newcommand{\FunctionTok}[1]{\textcolor[rgb]{0.28,0.35,0.67}{#1}}
\newcommand{\ImportTok}[1]{\textcolor[rgb]{0.00,0.46,0.62}{#1}}
\newcommand{\InformationTok}[1]{\textcolor[rgb]{0.37,0.37,0.37}{#1}}
\newcommand{\KeywordTok}[1]{\textcolor[rgb]{0.00,0.23,0.31}{#1}}
\newcommand{\NormalTok}[1]{\textcolor[rgb]{0.00,0.23,0.31}{#1}}
\newcommand{\OperatorTok}[1]{\textcolor[rgb]{0.37,0.37,0.37}{#1}}
\newcommand{\OtherTok}[1]{\textcolor[rgb]{0.00,0.23,0.31}{#1}}
\newcommand{\PreprocessorTok}[1]{\textcolor[rgb]{0.68,0.00,0.00}{#1}}
\newcommand{\RegionMarkerTok}[1]{\textcolor[rgb]{0.00,0.23,0.31}{#1}}
\newcommand{\SpecialCharTok}[1]{\textcolor[rgb]{0.37,0.37,0.37}{#1}}
\newcommand{\SpecialStringTok}[1]{\textcolor[rgb]{0.13,0.47,0.30}{#1}}
\newcommand{\StringTok}[1]{\textcolor[rgb]{0.13,0.47,0.30}{#1}}
\newcommand{\VariableTok}[1]{\textcolor[rgb]{0.07,0.07,0.07}{#1}}
\newcommand{\VerbatimStringTok}[1]{\textcolor[rgb]{0.13,0.47,0.30}{#1}}
\newcommand{\WarningTok}[1]{\textcolor[rgb]{0.37,0.37,0.37}{\textit{#1}}}

\providecommand{\tightlist}{%
  \setlength{\itemsep}{0pt}\setlength{\parskip}{0pt}}\usepackage{longtable,booktabs,array}
\usepackage{calc} % for calculating minipage widths
% Correct order of tables after \paragraph or \subparagraph
\usepackage{etoolbox}
\makeatletter
\patchcmd\longtable{\par}{\if@noskipsec\mbox{}\fi\par}{}{}
\makeatother
% Allow footnotes in longtable head/foot
\IfFileExists{footnotehyper.sty}{\usepackage{footnotehyper}}{\usepackage{footnote}}
\makesavenoteenv{longtable}
\usepackage{graphicx}
\makeatletter
\def\maxwidth{\ifdim\Gin@nat@width>\linewidth\linewidth\else\Gin@nat@width\fi}
\def\maxheight{\ifdim\Gin@nat@height>\textheight\textheight\else\Gin@nat@height\fi}
\makeatother
% Scale images if necessary, so that they will not overflow the page
% margins by default, and it is still possible to overwrite the defaults
% using explicit options in \includegraphics[width, height, ...]{}
\setkeys{Gin}{width=\maxwidth,height=\maxheight,keepaspectratio}
% Set default figure placement to htbp
\makeatletter
\def\fps@figure{htbp}
\makeatother

\usepackage{fontspec}
\usepackage{multirow}
\usepackage{multicol}
\usepackage{colortbl}
\usepackage{hhline}
\newlength\Oldarrayrulewidth
\newlength\Oldtabcolsep
\usepackage{longtable}
\usepackage{array}
\usepackage{hyperref}
\usepackage{float}
\usepackage{wrapfig}
\KOMAoption{captions}{tableheading}
\makeatletter
\makeatother
\makeatletter
\makeatother
\makeatletter
\@ifpackageloaded{caption}{}{\usepackage{caption}}
\AtBeginDocument{%
\ifdefined\contentsname
  \renewcommand*\contentsname{Table of contents}
\else
  \newcommand\contentsname{Table of contents}
\fi
\ifdefined\listfigurename
  \renewcommand*\listfigurename{List of Figures}
\else
  \newcommand\listfigurename{List of Figures}
\fi
\ifdefined\listtablename
  \renewcommand*\listtablename{List of Tables}
\else
  \newcommand\listtablename{List of Tables}
\fi
\ifdefined\figurename
  \renewcommand*\figurename{Figure}
\else
  \newcommand\figurename{Figure}
\fi
\ifdefined\tablename
  \renewcommand*\tablename{Table}
\else
  \newcommand\tablename{Table}
\fi
}
\@ifpackageloaded{float}{}{\usepackage{float}}
\floatstyle{ruled}
\@ifundefined{c@chapter}{\newfloat{codelisting}{h}{lop}}{\newfloat{codelisting}{h}{lop}[chapter]}
\floatname{codelisting}{Listing}
\newcommand*\listoflistings{\listof{codelisting}{List of Listings}}
\makeatother
\makeatletter
\@ifpackageloaded{caption}{}{\usepackage{caption}}
\@ifpackageloaded{subcaption}{}{\usepackage{subcaption}}
\makeatother
\makeatletter
\@ifpackageloaded{tcolorbox}{}{\usepackage[skins,breakable]{tcolorbox}}
\makeatother
\makeatletter
\@ifundefined{shadecolor}{\definecolor{shadecolor}{rgb}{.97, .97, .97}}
\makeatother
\makeatletter
\makeatother
\makeatletter
\makeatother
\ifLuaTeX
  \usepackage{selnolig}  % disable illegal ligatures
\fi
\IfFileExists{bookmark.sty}{\usepackage{bookmark}}{\usepackage{hyperref}}
\IfFileExists{xurl.sty}{\usepackage{xurl}}{} % add URL line breaks if available
\urlstyle{same} % disable monospaced font for URLs
\hypersetup{
  pdftitle={Homework 4},
  pdfauthor={Kyle Cahitas},
  colorlinks=true,
  linkcolor={blue},
  filecolor={Maroon},
  citecolor={Blue},
  urlcolor={Blue},
  pdfcreator={LaTeX via pandoc}}

\title{Homework 4}
\author{Kyle Cahitas}
\date{}

\begin{document}
\maketitle
\ifdefined\Shaded\renewenvironment{Shaded}{\begin{tcolorbox}[breakable, enhanced, sharp corners, borderline west={3pt}{0pt}{shadecolor}, boxrule=0pt, interior hidden, frame hidden]}{\end{tcolorbox}}\fi

\hypertarget{question}{%
\subsection{Question}\label{question}}

\begin{Shaded}
\begin{Highlighting}[]
\FunctionTok{library}\NormalTok{(tidyverse)}
\end{Highlighting}
\end{Shaded}

\begin{verbatim}
-- Attaching core tidyverse packages ------------------------ tidyverse 2.0.0 --
v dplyr     1.1.1     v readr     2.1.4
v forcats   1.0.0     v stringr   1.5.0
v ggplot2   3.4.2     v tibble    3.2.1
v lubridate 1.9.2     v tidyr     1.3.0
v purrr     1.0.1     
-- Conflicts ------------------------------------------ tidyverse_conflicts() --
x dplyr::filter() masks stats::filter()
x dplyr::lag()    masks stats::lag()
i Use the conflicted package (<http://conflicted.r-lib.org/>) to force all conflicts to become errors
\end{verbatim}

\begin{Shaded}
\begin{Highlighting}[]
\FunctionTok{library}\NormalTok{(ggplot2)}
\FunctionTok{library}\NormalTok{(dbplyr)}
\end{Highlighting}
\end{Shaded}

\begin{verbatim}

Attaching package: 'dbplyr'

The following objects are masked from 'package:dplyr':

    ident, sql
\end{verbatim}

\begin{Shaded}
\begin{Highlighting}[]
\FunctionTok{library}\NormalTok{(here)}
\end{Highlighting}
\end{Shaded}

\begin{verbatim}
here() starts at /Users/kyleyahitas/Desktop/github/ENVS-193DS_homework-04_cahitas_kyle
\end{verbatim}

\begin{Shaded}
\begin{Highlighting}[]
\FunctionTok{library}\NormalTok{(janitor)}
\end{Highlighting}
\end{Shaded}

\begin{verbatim}

Attaching package: 'janitor'

The following objects are masked from 'package:stats':

    chisq.test, fisher.test
\end{verbatim}

\begin{Shaded}
\begin{Highlighting}[]
\FunctionTok{library}\NormalTok{(naniar)}
\FunctionTok{library}\NormalTok{(skimr)}
\end{Highlighting}
\end{Shaded}

\begin{verbatim}

Attaching package: 'skimr'

The following object is masked from 'package:naniar':

    n_complete
\end{verbatim}

\begin{Shaded}
\begin{Highlighting}[]
\FunctionTok{library}\NormalTok{(knitr)}
\FunctionTok{library}\NormalTok{(ggeffects)}
\FunctionTok{library}\NormalTok{(performance)}
\FunctionTok{library}\NormalTok{(broom)}
\FunctionTok{library}\NormalTok{(flextable)}
\end{Highlighting}
\end{Shaded}

\begin{verbatim}

Attaching package: 'flextable'

The following object is masked from 'package:purrr':

    compose
\end{verbatim}

\begin{Shaded}
\begin{Highlighting}[]
\FunctionTok{library}\NormalTok{(ggeffects)}
\FunctionTok{library}\NormalTok{(car)}
\end{Highlighting}
\end{Shaded}

\begin{verbatim}
Loading required package: carData

Attaching package: 'car'

The following object is masked from 'package:dplyr':

    recode

The following object is masked from 'package:purrr':

    some
\end{verbatim}

\begin{Shaded}
\begin{Highlighting}[]
\CommentTok{\#filtering data, importing csv}
\NormalTok{fish\_df }\OtherTok{\textless{}{-}} \FunctionTok{read\_csv}\NormalTok{(}\FunctionTok{here}\NormalTok{(}\StringTok{"data"}\NormalTok{, }\StringTok{"ntl6\_v12.csv"}\NormalTok{)) }\SpecialCharTok{\%\textgreater{}\%} 
  \FunctionTok{filter}\NormalTok{(spname }\SpecialCharTok{==} \StringTok{"TROUTPERCH"}\NormalTok{) }\SpecialCharTok{\%\textgreater{}\%} 
  \FunctionTok{select}\NormalTok{(spname,year4,length,weight) }\SpecialCharTok{\%\textgreater{}\%} 
  \FunctionTok{rename}\NormalTok{(}\AttributeTok{year =}\NormalTok{ year4)}
\end{Highlighting}
\end{Shaded}

\begin{verbatim}
Warning: One or more parsing issues, call `problems()` on your data frame for details,
e.g.:
  dat <- vroom(...)
  problems(dat)
\end{verbatim}

\begin{verbatim}
Rows: 349229 Columns: 15
-- Column specification --------------------------------------------------------
Delimiter: ","
chr  (8): lakeid, gearid, spname, sampletype, indid, fishpart, spseq, flag
dbl  (5): year4, depth, rep, length, weight
lgl  (1): sex
date (1): sampledate

i Use `spec()` to retrieve the full column specification for this data.
i Specify the column types or set `show_col_types = FALSE` to quiet this message.
\end{verbatim}

\begin{Shaded}
\begin{Highlighting}[]
\FunctionTok{View}\NormalTok{(fish\_df)}
\end{Highlighting}
\end{Shaded}

How does fish length predict fish weight for trout perch (across all
sample years)?

Null Hypothesis

Alternative Hypothesis

\begin{Shaded}
\begin{Highlighting}[]
\CommentTok{\# 200 missing rows for weight}
\CommentTok{\# have all rows for length }
\FunctionTok{gg\_miss\_var}\NormalTok{(fish\_df)}
\end{Highlighting}
\end{Shaded}

\begin{figure}[H]

{\centering \includegraphics{homework-4_files/figure-pdf/missing-data-vis-1.pdf}

}

\end{figure}

\begin{Shaded}
\begin{Highlighting}[]
\CommentTok{\#using tidyverse to filter out any Na values in the weight column }
\NormalTok{troutperch\_Na }\OtherTok{\textless{}{-}}\NormalTok{ fish\_df }\SpecialCharTok{\%\textgreater{}\%} 
  \FunctionTok{drop\_na}\NormalTok{(weight)}
\FunctionTok{View}\NormalTok{(troutperch\_Na)}
\end{Highlighting}
\end{Shaded}

\begin{Shaded}
\begin{Highlighting}[]
\FunctionTok{ggplot}\NormalTok{(}\AttributeTok{data =}\NormalTok{ troutperch\_Na, }\FunctionTok{aes}\NormalTok{(}\AttributeTok{x =}\NormalTok{ length, }\AttributeTok{y =}\NormalTok{ weight)) }\SpecialCharTok{+} 
  
  \FunctionTok{geom\_point}\NormalTok{()}\SpecialCharTok{+}
  
  \FunctionTok{theme\_classic}\NormalTok{()}
\end{Highlighting}
\end{Shaded}

\begin{figure}[H]

{\centering \includegraphics[width=0.8\textwidth,height=\textheight]{homework-4_files/figure-pdf/unnamed-chunk-1-1.pdf}

}

\end{figure}

\begin{Shaded}
\begin{Highlighting}[]
\NormalTok{modelobject }\OtherTok{\textless{}{-}} \FunctionTok{lm}\NormalTok{(weight }\SpecialCharTok{\textasciitilde{}}\NormalTok{ length, }\AttributeTok{data =}\NormalTok{ troutperch\_Na)}
\FunctionTok{par}\NormalTok{(}\AttributeTok{mfrow =} \FunctionTok{c}\NormalTok{(}\DecValTok{2}\NormalTok{,}\DecValTok{2}\NormalTok{))}
\FunctionTok{plot}\NormalTok{(modelobject)}
\end{Highlighting}
\end{Shaded}

\begin{figure}[H]

{\centering \includegraphics[width=0.8\textwidth,height=\textheight]{homework-4_files/figure-pdf/unnamed-chunk-2-1.pdf}

}

\end{figure}

\begin{Shaded}
\begin{Highlighting}[]
\DocumentationTok{\#\#\#\#\#}
\CommentTok{\# when looking at the Residual vs Fitted, it looks Heteroscedasticity }
\CommentTok{\# Normal Q{-}Q, the data looks normal }
\CommentTok{\# scale {-}location, is the same thing like Residuals vs fitted }
\CommentTok{\# Residuals vs leverage }
\end{Highlighting}
\end{Shaded}

\begin{Shaded}
\begin{Highlighting}[]
\NormalTok{predictions }\OtherTok{\textless{}{-}} \FunctionTok{ggpredict}\NormalTok{(modelobject, }\AttributeTok{terms =} \StringTok{"length"}\NormalTok{)}

\FunctionTok{View}\NormalTok{(predictions)}
\end{Highlighting}
\end{Shaded}

\begin{Shaded}
\begin{Highlighting}[]
\NormalTok{plot\_predictions }\OtherTok{\textless{}{-}} \FunctionTok{ggplot}\NormalTok{(}\AttributeTok{data =}\NormalTok{ troutperch\_Na, }
                           \FunctionTok{aes}\NormalTok{(}\AttributeTok{x =}\NormalTok{ length, }\AttributeTok{y =}\NormalTok{ weight)) }\SpecialCharTok{+}
  \CommentTok{\# first plot the underlying data from maples\_data}
  \FunctionTok{geom\_point}\NormalTok{() }\SpecialCharTok{+}
  \CommentTok{\# then plot the predictions}
  \FunctionTok{geom\_line}\NormalTok{(}\AttributeTok{data =}\NormalTok{ predictions, }
            \FunctionTok{aes}\NormalTok{(}\AttributeTok{x =}\NormalTok{ x, }\AttributeTok{y =}\NormalTok{ predicted), }
            \AttributeTok{color =} \StringTok{"blue"}\NormalTok{, }\AttributeTok{linewidth =} \DecValTok{1}\NormalTok{) }\SpecialCharTok{+}
  \CommentTok{\# then plot the 95\% confidence interval from ggpredict}
  \FunctionTok{geom\_ribbon}\NormalTok{(}\AttributeTok{data =}\NormalTok{ predictions, }
              \FunctionTok{aes}\NormalTok{(}\AttributeTok{x =}\NormalTok{ x, }\AttributeTok{y =}\NormalTok{ predicted, }\AttributeTok{ymin =}\NormalTok{ conf.low, }\AttributeTok{ymax =}\NormalTok{ conf.high), }
              \AttributeTok{alpha =} \FloatTok{0.2}\NormalTok{) }\SpecialCharTok{+}
  \CommentTok{\# theme and meaningful labels}
  \FunctionTok{theme\_bw}\NormalTok{() }\SpecialCharTok{+}
 \FunctionTok{labs}\NormalTok{(}\AttributeTok{x =} \StringTok{"Stem length (mm)"}\NormalTok{,}
       \AttributeTok{y =} \StringTok{"Stem dry mass (g)"}\NormalTok{)}

\NormalTok{plot\_predictions}
\end{Highlighting}
\end{Shaded}

\begin{figure}[H]

{\centering \includegraphics{homework-4_files/figure-pdf/unnamed-chunk-4-1.pdf}

}

\end{figure}

\begin{Shaded}
\begin{Highlighting}[]
\NormalTok{model\_summary }\OtherTok{\textless{}{-}} \FunctionTok{summary}\NormalTok{(modelobject)}

\NormalTok{model\_sqaures }\OtherTok{\textless{}{-}} \FunctionTok{anova}\NormalTok{(modelobject)}

\FunctionTok{View}\NormalTok{(model\_sqaures)}
\end{Highlighting}
\end{Shaded}

\begin{Shaded}
\begin{Highlighting}[]
\NormalTok{model\_squares\_table }\OtherTok{\textless{}{-}}\FunctionTok{tidy}\NormalTok{(model\_sqaures) }\SpecialCharTok{\%\textgreater{}\%} 
  
  \FunctionTok{mutate}\NormalTok{(}\AttributeTok{p.value =} \FunctionTok{case\_when}\NormalTok{(p.value }\SpecialCharTok{\textless{}}\FloatTok{0.001} \SpecialCharTok{\textasciitilde{}} \StringTok{"\textless{}0.001"}\NormalTok{)) }\SpecialCharTok{\%\textgreater{}\%} 
  
  \FunctionTok{flextable}\NormalTok{() }\SpecialCharTok{\%\textgreater{}\%} 
  \CommentTok{\#need to change the weight name to luike fis}
  \FunctionTok{set\_header\_labels}\NormalTok{(}\AttributeTok{df =} \StringTok{"Degrees of Freedom"}\NormalTok{, }
                    \AttributeTok{sumsq =} \StringTok{"Sum of squares"}\NormalTok{,}
                    \AttributeTok{meansq =} \StringTok{"Mean squares"}\NormalTok{,}
                    \AttributeTok{statistic =} \StringTok{"F{-}statistic"}\NormalTok{,}
                    \AttributeTok{p.value =} \StringTok{"p{-}value"}
\NormalTok{                    )}
\NormalTok{model\_squares\_table}
\end{Highlighting}
\end{Shaded}

\global\setlength{\Oldarrayrulewidth}{\arrayrulewidth}

\global\setlength{\Oldtabcolsep}{\tabcolsep}

\setlength{\tabcolsep}{0pt}

\renewcommand*{\arraystretch}{1.5}



\providecommand{\ascline}[3]{\noalign{\global\arrayrulewidth #1}\arrayrulecolor[HTML]{#2}\cline{#3}}

\begin{longtable*}[c]{|p{0.75in}|p{0.75in}|p{0.75in}|p{0.75in}|p{0.75in}|p{0.75in}}



\ascline{1.5pt}{666666}{1-6}

\multicolumn{1}{>{\raggedright}m{\dimexpr 0.75in+0\tabcolsep}}{\textcolor[HTML]{000000}{\fontsize{11}{11}\selectfont{term}}} & \multicolumn{1}{>{\raggedleft}m{\dimexpr 0.75in+0\tabcolsep}}{\textcolor[HTML]{000000}{\fontsize{11}{11}\selectfont{Degrees\ of\ Freedom}}} & \multicolumn{1}{>{\raggedleft}m{\dimexpr 0.75in+0\tabcolsep}}{\textcolor[HTML]{000000}{\fontsize{11}{11}\selectfont{Sum\ of\ squares}}} & \multicolumn{1}{>{\raggedleft}m{\dimexpr 0.75in+0\tabcolsep}}{\textcolor[HTML]{000000}{\fontsize{11}{11}\selectfont{Mean\ squares}}} & \multicolumn{1}{>{\raggedleft}m{\dimexpr 0.75in+0\tabcolsep}}{\textcolor[HTML]{000000}{\fontsize{11}{11}\selectfont{F-statistic}}} & \multicolumn{1}{>{\raggedright}m{\dimexpr 0.75in+0\tabcolsep}}{\textcolor[HTML]{000000}{\fontsize{11}{11}\selectfont{p-value}}} \\

\ascline{1.5pt}{666666}{1-6}\endhead



\multicolumn{1}{>{\raggedright}m{\dimexpr 0.75in+0\tabcolsep}}{\textcolor[HTML]{000000}{\fontsize{11}{11}\selectfont{length}}} & \multicolumn{1}{>{\raggedleft}m{\dimexpr 0.75in+0\tabcolsep}}{\textcolor[HTML]{000000}{\fontsize{11}{11}\selectfont{1}}} & \multicolumn{1}{>{\raggedleft}m{\dimexpr 0.75in+0\tabcolsep}}{\textcolor[HTML]{000000}{\fontsize{11}{11}\selectfont{1,432.2877}}} & \multicolumn{1}{>{\raggedleft}m{\dimexpr 0.75in+0\tabcolsep}}{\textcolor[HTML]{000000}{\fontsize{11}{11}\selectfont{1,432.287687}}} & \multicolumn{1}{>{\raggedleft}m{\dimexpr 0.75in+0\tabcolsep}}{\textcolor[HTML]{000000}{\fontsize{11}{11}\selectfont{1,280.844}}} & \multicolumn{1}{>{\raggedright}m{\dimexpr 0.75in+0\tabcolsep}}{\textcolor[HTML]{000000}{\fontsize{11}{11}\selectfont{<0.001}}} \\





\multicolumn{1}{>{\raggedright}m{\dimexpr 0.75in+0\tabcolsep}}{\textcolor[HTML]{000000}{\fontsize{11}{11}\selectfont{Residuals}}} & \multicolumn{1}{>{\raggedleft}m{\dimexpr 0.75in+0\tabcolsep}}{\textcolor[HTML]{000000}{\fontsize{11}{11}\selectfont{288}}} & \multicolumn{1}{>{\raggedleft}m{\dimexpr 0.75in+0\tabcolsep}}{\textcolor[HTML]{000000}{\fontsize{11}{11}\selectfont{322.0525}}} & \multicolumn{1}{>{\raggedleft}m{\dimexpr 0.75in+0\tabcolsep}}{\textcolor[HTML]{000000}{\fontsize{11}{11}\selectfont{1.118238}}} & \multicolumn{1}{>{\raggedleft}m{\dimexpr 0.75in+0\tabcolsep}}{\textcolor[HTML]{000000}{\fontsize{11}{11}\selectfont{}}} & \multicolumn{1}{>{\raggedright}m{\dimexpr 0.75in+0\tabcolsep}}{\textcolor[HTML]{000000}{\fontsize{11}{11}\selectfont{}}} \\

\ascline{1.5pt}{666666}{1-6}



\end{longtable*}



\arrayrulecolor[HTML]{000000}

\global\setlength{\arrayrulewidth}{\Oldarrayrulewidth}

\global\setlength{\tabcolsep}{\Oldtabcolsep}

\renewcommand*{\arraystretch}{1}



\end{document}
